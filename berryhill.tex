
% Template for Elsevier CRC journal article
% version 1.1 dated 16 March 2010

% This file (c) 2009-10 Elsevier Ltd.  Modifications may be freely made,
% provided the edited file is saved under a different name

% This file contains modifications for Nuclear Physics B Proceedings Supplement

% Changes since version 1.0
% - elsarticle class option changed from 1p to 3p (to better reflect CRC layout)
%

%-----------------------------------------------------------------------------------

%% This template uses the elsarticle.cls document class and the extension package ecrc.sty
%% For full documentation on usage of elsarticle.cls, consult the documentation "elsdoc.pdf"
%% Further resources available at http://www.elsevier.com/latex

%-----------------------------------------------------------------------------------

%%%%%%%%%%%%%%%%%%%%%%%%%%%%%%%%%%%%%%%%%%%%%%
%%%%%%%%%%%%%%%%%%%%%%%%%%%%%%%%%%%%%%%%%%%%%%
%%                                          %%
%% Important note on usage                  %%
%% -----------------------                  %%
%% This file must be compiled with PDFLaTeX %%
%% Using standard LaTeX will not work!      %%
%%                                          %%
%%%%%%%%%%%%%%%%%%%%%%%%%%%%%%%%%%%%%%%%%%%%%%
%%%%%%%%%%%%%%%%%%%%%%%%%%%%%%%%%%%%%%%%%%%%%%

%% The '3p' and 'times' class options of elsarticle are used for Elsevier CRC
\documentclass[3p,times,twocolumn]{elsarticle}

%% The `ecrc' package must be called to make the CRC functionality available
\usepackage{ecrc}

%% The ecrc package defines commands needed for running heads and logos.
%% For running heads, you can set the journal name, the volume, the starting page and the authors

%% set the volume if you know. Otherwise `00'
\volume{00}

%% set the starting page if not 1
\firstpage{1}

%% Give the name of the journal
\journalname{Nuclear Physics B Proceedings Supplement}

%% Give the author list to appear in the running head
%% Example \runauth{C.V. Radhakrishnan et al.}
\runauth{J. Berryhill}

%% The choice of journal logo is determined by the \jid and \jnltitlelogo commands.
%% A user-supplied logo with the name <\jid>logo.pdf will be inserted if present.
%% e.g. if \jid{yspmi} the system will look for a file yspmilogo.pdf
%% Otherwise the content of \jnltitlelogo will be set between horizontal lines as a default logo

%% Give the abbreviation of the Journal.
\jid{nuphbp}

%% Give a short journal name for the dummy logo (if needed)
\jnltitlelogo{Nuclear Physics B Proceedings Supplement}

%% Hereafter the template follows `elsarticle'.
%% For more details see the existing template files elsarticle-template-harv.tex and elsarticle-template-num.tex.

%% Elsevier CRC generally uses a numbered reference style
%% For this, the conventions of elsarticle-template-num.tex should be followed (included below)
%% If using BibTeX, use the style file elsarticle-num.bst

%% End of ecrc-specific commands
%%%%%%%%%%%%%%%%%%%%%%%%%%%%%%%%%%%%%%%%%%%%%%%%%%%%%%%%%%%%%%%%%%%%%%%%%%

%% The amssymb package provides various useful mathematical symbols
\usepackage{amssymb}
%% The amsthm package provides extended theorem environments
%% \usepackage{amsthm}

%% The lineno packages adds line numbers. Start line numbering with
%% \begin{linenumbers}, end it with \end{linenumbers}. Or switch it on
%% for the whole article with \linenumbers after \end{frontmatter}.
%% \usepackage{lineno}

%% natbib.sty is loaded by default. However, natbib options can be
%% provided with \biboptions{...} command. Following options are
%% valid:

%%   round  -  round parentheses are used (default)
%%   square -  square brackets are used   [option]
%%   curly  -  curly braces are used      {option}
%%   angle  -  angle brackets are used    <option>
%%   semicolon  -  multiple citations separated by semi-colon
%%   colon  - same as semicolon, an earlier confusion
%%   comma  -  separated by comma
%%   numbers-  selects numerical citations
%%   super  -  numerical citations as superscripts
%%   sort   -  sorts multiple citations according to order in ref. list
%%   sort&compress   -  like sort, but also compresses numerical citations
%%   compress - compresses without sorting
%%
%% \biboptions{comma,round}

% \biboptions{}

% if you have landscape tables
\usepackage[figuresright]{rotating}

% put your own definitions here:
%   \newcommand{\cZ}{\cal{Z}}
%   \newtheorem{def}{Definition}[section]
%   ...

% add words to TeX's hyphenation exception list
%\hyphenation{author another created financial paper re-commend-ed Post-Script}

% declarations for front matter

\begin{document}

\begin{frontmatter}

%% Title, authors and addresses

%% use the tnoteref command within \title for footnotes;
%% use the tnotetext command for the associated footnote;
%% use the fnref command within \author or \address for footnotes;
%% use the fntext command for the associated footnote;
%% use the corref command within \author for corresponding author footnotes;
%% use the cortext command for the associated footnote;
%% use the ead command for the email address,
%% and the form \ead[url] for the home page:
%%
%% \title{Title\tnoteref{label1}}
%% \tnotetext[label1]{}
%% \author{Name\corref{cor1}\fnref{label2}}
%% \ead{email address}
%% \ead[url]{home page}
%% \fntext[label2]{}
%% \cortext[cor1]{}
%% \address{Address\fnref{label3}}
%% \fntext[label3]{}

\dochead{}
%% Use \dochead if there is an article header, e.g. \dochead{Short communication}

\title{Tests of the Electroweak Interactions at Hadron Colliders}

%% use optional labels to link authors explicitly to addresses:
%% \author[label1,label2]{<author name>}
%% \address[label1]{<address>}
%% \address[label2]{<address>}

\author{Jeffrey Berryhill, on behalf of the ATLAS, CDF, CMS, D0, and LHCb collaborations}

\address{Fermi National Accelerator Laboratory (FNAL), Batavia, IL 60510, USA}

\begin{abstract}
%% Text of abstract
\end{abstract}

\begin{keyword}
%% keywords here, in the form: keyword \sep keyword

%% MSC codes here, in the form: \MSC code \sep code
%% or \MSC[2008] code \sep code (2000 is the default)

\end{keyword}

\end{frontmatter}

%%
%% Start line numbering here if you want
%%
% \linenumbers

%% main text
\section{$W$ boson physics}
\label{Wboson}

ATLAS~\cite{Aad:2008zzm}
CDF~\cite{Abulencia:2005ix}
CMS~\cite{CMSdetector}
D0~\cite{Abazov:2005pn}
LHCb~\cite{Alves:2008zz}

ATLAS high-mass Drell--Yan~\cite{Aad:2013iua} 
ATLAS low-mass Drell-Yan~\cite{Aad:2014qja}
ATLAS Z PT~\cite{Aad:2014xaa}
ATLAS Z phistar~\cite{Aad:2012wfa}
ATLAS Wgamma Zgamma~\cite{Aad:2013izg}
ATLAS WW 7 TeV~\cite{ATLAS:2012mec}
ATLAS WW 8 TeV~\cite{ATLAS-CONF-2014-033}

CDF Z asymmetry muon~\cite{Aaltonen:2014loa}
CDF Z asymmetry electron~\cite{Aaltonen:2013wcp}
CDF W mass PRD~\cite{Aaltonen:2013vwa}
CDF W mass PRL~\cite{Aaltonen:2012bp}

CMS Drell--Yan 7 TeV~\cite{Chatrchyan:2013tia}
CMS W asymmetry muon~\cite{Chatrchyan:2013mza}
CMS W asymmetry electron~\cite{Chatrchyan:2012xt}
CMS W+charm~\cite{Chatrchyan:2013uja}
CMS ZZ4l 8 TeV~\cite{Khachatryan:2014dia}
CMS ZZ4l 7 TeV~\cite{Chatrchyan:2012sga}
CMS WW/ZZ 8 TeV~\cite{Chatrchyan:2013oev}
CMS WW2l2n 7 TeV~\cite{Chatrchyan:2013yaa}
CMS WWlnjj 7 TeV~\cite{Chatrchyan:2012bd}
CMS WVgamma 8 TeV~\cite{Chatrchyan:2014bza}
CMS Wgamma/Zgamma 7 TeV~\cite{Chatrchyan:2013fya}
CMS Znngamma 7 TeV~\cite{Chatrchyan:2013nda}
CMS WWexcl 7 TeV~\cite{Chatrchyan:2013foa}
CMS VBF Z 7 TeV~\cite{Chatrchyan:2013jya}
CMS SSWW 8 TeV~\cite{CMS-PAS-SMP-13-015}

D0 W asymmetry electron~\cite{Abazov:2013dsa}
D0 W asymmetry muon~\cite{Abazov:2013rja}
D0 W mass PRD~\cite{D0:2013jba}
D0 W mass PRL~\cite{Abazov:2012bv}

CDF+D0 W mass combination~\cite{Aaltonen:2013iut}

Snowmass electroweak~\cite{Baak:2013fwa}

Wmass PDF~\cite{Bozzi:2011ww}

ATLAS WW scattering~\cite{Aad:2014zda}
ATLAS VBF Z~\cite{Aad:2014dta}


\section{$Z$ boson and Drell-Yan production physics}
\label{ZDY}

\section{Triple gauge couplings}
\label{TGC}

\section{Quartic gauge couplings}
\label{QGC}
%% The Appendices part is started with the command \appendix;
%% appendix sections are then done as normal sections
%% \appendix

%% \section{}
%% \label{}

%% References
%%
%% Following citation commands can be used in the body text:
%% Usage of \cite is as follows:
%%   \cite{key}         ==>>  [#]
%%   \cite[chap. 2]{key} ==>> [#, chap. 2]
%%

%% References with BibTeX database:
\nocite{*}
\bibliographystyle{elsarticle-num}
\bibliography{berryhill}

%% Authors are advised to use a BibTeX database file for their reference list.
%% The provided style file elsarticle-num.bst formats references in the required Procedia style

%% For references without a BibTeX database:

% \begin{thebibliography}{00}

%% \bibitem must have the following form:
%%   \bibitem{key}...
%%

% \bibitem{}

% \end{thebibliography}

\end{document}

%%
%% End of file `nuphbp-template.tex'. 
